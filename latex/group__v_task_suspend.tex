\hypertarget{group__v_task_suspend}{}\section{v\+Task\+Suspend}
\label{group__v_task_suspend}\index{v\+Task\+Suspend@{v\+Task\+Suspend}}
task. h 
\begin{DoxyPre}void vTaskSuspend( TaskHandle\_t xTaskToSuspend );\end{DoxyPre}


I\+N\+C\+L\+U\+D\+E\+\_\+v\+Task\+Suspend must be defined as 1 for this function to be available. See the configuration section for more information.

Suspend any task. When suspended a task will never get any microcontroller processing time, no matter what its priority.

Calls to v\+Task\+Suspend are not accumulative -\/ i.\+e. calling v\+Task\+Suspend () twice on the same task still only requires one call to v\+Task\+Resume () to ready the suspended task.


\begin{DoxyParams}{Parameters}
{\em x\+Task\+To\+Suspend} & Handle to the task being suspended. Passing a N\+U\+L\+L handle will cause the calling task to be suspended.\\
\hline
\end{DoxyParams}
Example usage\+: 
\begin{DoxyPre}
void vAFunction( void )
\{
TaskHandle\_t xHandle;
\begin{DoxyVerb}// Create a task, storing the handle.
xTaskCreate( vTaskCode, "NAME", STACK_SIZE, NULL, tskIDLE_PRIORITY, &xHandle );

// ...

// Use the handle to suspend the created task.
vTaskSuspend( xHandle );

// ...

// The created task will not run during this period, unless
// another task calls vTaskResume( xHandle ).

//...


// Suspend ourselves.
vTaskSuspend( NULL );

// We cannot get here unless another task calls vTaskResume
// with our handle as the parameter.
\end{DoxyVerb}

\}
  \end{DoxyPre}
 